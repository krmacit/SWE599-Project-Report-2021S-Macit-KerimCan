
% %% bare_jrnl_comsoc.tex
% %% V1.4b
% %% 2015/08/26
% %% by Michael Shell
% %% see http://www.michaelshell.org/
% %% for current contact information.
% %%
% %% This is a skeleton file demonstrating the use of IEEEtran.cls
% %% (requires IEEEtran.cls version 1.8b or later) with an IEEE
% %% Communications Society journal paper.
% %%
% %% Support sites:
% %% http://www.michaelshell.org/tex/ieeetran/
% %% http://www.ctan.org/pkg/ieeetran
% %% and
% %% http://www.ieee.org/

% \documentclass[journal,comsoc]{IEEEtran}
% %
% % If IEEEtran.cls has not been installed into the LaTeX system files,
% % manually specify the path to it like:
% % \documentclass[journal,comsoc]{../sty/IEEEtran}


% \usepackage[T1]{fontenc}% optional T1 font encoding






% % *** CITATION PACKAGES ***
% %
% \usepackage{cite}

% % cite.sty was written by Donald Arseneau
% % V1.6 and later of IEEEtran pre-defines the format of the cite.sty package
% % \cite{} output to follow that of the IEEE. Loading the cite package will
% % result in citation numbers being automatically sorted and properly
% % "compressed/ranged". e.g., [1], [9], [2], [7], [5], [6] without using
% % cite.sty will become [1], [2], [5]--[7], [9] using cite.sty. cite.sty's
% % \cite will automatically add leading space, if needed. Use cite.sty's
% % noadjust option (cite.sty V3.8 and later) if you want to turn this off
% % such as if a citation ever needs to be enclosed in parenthesis.
% % cite.sty is already installed on most LaTeX systems. Be sure and use
% % version 5.0 (2009-03-20) and later if using hyperref.sty.
% % The latest version can be obtained at:
% % http://www.ctan.org/pkg/cite
% % The documentation is contained in the cite.sty file itself.





% % *** GRAPHICS RELATED PACKAGES ***
% \usepackage[pdftex]{graphicx}
% \usepackage{epstopdf}

% %
% %\ifCLASSINFOpdf
% %  \usepackage[pdftex]{graphicx}
% %  % declare the path(s) where your graphic files are
% %  % \graphicspath{{../pdf/}{../jpeg/}}
% %  % and their extensions so you won't have to specify these with
% %  % every instance of \includegraphics
% %  % \DeclareGraphicsExtensions{.pdf,.jpeg,.png}
% %\else
% %  % or other class option (dvipsone, dvipdf, if not using dvips). graphicx
% %  % will default to the driver specified in the system graphics.cfg if no
% %  % driver is specified.
% %  % \usepackage[dvips]{graphicx}
% %  % declare the path(s) where your graphic files are
% %  % \graphicspath{{../eps/}}
% %  % and their extensions so you won't have to specify these with
% %  % every instance of \includegraphics
% %  % \DeclareGraphicsExtensions{.eps}
% %\fi
% % graphicx was written by David Carlisle and Sebastian Rahtz. It is
% % required if you want graphics, photos, etc. graphicx.sty is already
% % installed on most LaTeX systems. The latest version and documentation
% % can be obtained at: 
% % http://www.ctan.org/pkg/graphicx
% % Another good source of documentation is "Using Imported Graphics in
% % LaTeX2e" by Keith Reckdahl which can be found at:
% % http://www.ctan.org/pkg/epslatex
% %
% % latex, and pdflatex in dvi mode, support graphics in encapsulated
% % postscript (.eps) format. pdflatex in pdf mode supports graphics
% % in .pdf, .jpeg, .png and .mps (metapost) formats. Users should ensure
% % that all non-photo figures use a vector format (.eps, .pdf, .mps) and
% % not a bitmapped formats (.jpeg, .png). The IEEE frowns on bitmapped formats
% % which can result in "jaggedy"/blurry rendering of lines and letters as
% % well as large increases in file sizes.
% %
% % You can find documentation about the pdfTeX application at:
% % http://www.tug.org/applications/pdftex





% % *** MATH PACKAGES ***
% %
% \usepackage{amsmath}

% %\usepackage{bm}
% % The bm.sty package was written by David Carlisle and Frank Mittelbach.
% % This package provides a \bm{} to produce bold math symbols.
% % http://www.ctan.org/pkg/bm





% % *** SPECIALIZED LIST PACKAGES ***
% %
% %\usepackage{algorithmic}
% % algorithmic.sty was written by Peter Williams and Rogerio Brito.
% % This package provides an algorithmic environment fo describing algorithms.
% % You can use the algorithmic environment in-text or within a figure
% % environment to provide for a floating algorithm. Do NOT use the algorithm
% % floating environment provided by algorithm.sty (by the same authors) or
% % algorithm2e.sty (by Christophe Fiorio) as the IEEE does not use dedicated
% % algorithm float types and packages that provide these will not provide
% % correct IEEE style captions. The latest version and documentation of
% % algorithmic.sty can be obtained at:
% % http://www.ctan.org/pkg/algorithms
% % Also of interest may be the (relatively newer and more customizable)
% % algorithmicx.sty package by Szasz Janos:
% % http://www.ctan.org/pkg/algorithmicx




% % *** ALIGNMENT PACKAGES ***
% %
% %\usepackage{array}
% % Frank Mittelbach's and David Carlisle's array.sty patches and improves
% % the standard LaTeX2e array and tabular environments to provide better
% % appearance and additional user controls. As the default LaTeX2e table
% % generation code is lacking to the point of almost being broken with
% % respect to the quality of the end results, all users are strongly
% % advised to use an enhanced (at the very least that provided by array.sty)
% % set of table tools. array.sty is already installed on most systems. The
% % latest version and documentation can be obtained at:
% % http://www.ctan.org/pkg/array


% % IEEEtran contains the IEEEeqnarray family of commands that can be used to
% % generate multiline equations as well as matrices, tables, etc., of high
% % quality.




% % *** SUBFIGURE PACKAGES ***
% \usepackage[caption=false,font=footnotesize]{subfig}
% %\ifCLASSOPTIONcompsoc
% %  \usepackage[caption=false,font=normalsize,labelfont=sf,textfont=sf]{subfig}
% %\else
% %  \usepackage[caption=false,font=footnotesize]{subfig}
% %\fi
% % subfig.sty, written by Steven Douglas Cochran, is the modern replacement
% % for subfigure.sty, the latter of which is no longer maintained and is
% % incompatible with some LaTeX packages including fixltx2e. However,
% % subfig.sty requires and automatically loads Axel Sommerfeldt's caption.sty
% % which will override IEEEtran.cls' handling of captions and this will result
% % in non-IEEE style figure/table captions. To prevent this problem, be sure
% % and invoke subfig.sty's "caption=false" package option (available since
% % subfig.sty version 1.3, 2005/06/28) as this is will preserve IEEEtran.cls
% % handling of captions.
% % Note that the Computer Society format requires a larger sans serif font
% % than the serif footnote size font used in traditional IEEE formatting
% % and thus the need to invoke different subfig.sty package options depending
% % on whether compsoc mode has been enabled.
% %
% % The latest version and documentation of subfig.sty can be obtained at:
% % http://www.ctan.org/pkg/subfig




% % *** FLOAT PACKAGES ***
% %
% %\usepackage{fixltx2e}
% % fixltx2e, the successor to the earlier fix2col.sty, was written by
% % Frank Mittelbach and David Carlisle. This package corrects a few problems
% % in the LaTeX2e kernel, the most notable of which is that in current
% % LaTeX2e releases, the ordering of single and double column floats is not
% % guaranteed to be preserved. Thus, an unpatched LaTeX2e can allow a
% % single column figure to be placed prior to an earlier double column
% % figure.
% % Be aware that LaTeX2e kernels dated 2015 and later have fixltx2e.sty's
% % corrections already built into the system in which case a warning will
% % be issued if an attempt is made to load fixltx2e.sty as it is no longer
% % needed.
% % The latest version and documentation can be found at:
% % http://www.ctan.org/pkg/fixltx2e


% %\usepackage{stfloats}
% % stfloats.sty was written by Sigitas Tolusis. This package gives LaTeX2e
% % the ability to do double column floats at the bottom of the page as well
% % as the top. (e.g., "\begin{figure*}[!b]" is not normally possible in
% % LaTeX2e). It also provides a command:
% %\fnbelowfloat
% % to enable the placement of footnotes below bottom floats (the standard
% % LaTeX2e kernel puts them above bottom floats). This is an invasive package
% % which rewrites many portions of the LaTeX2e float routines. It may not work
% % with other packages that modify the LaTeX2e float routines. The latest
% % version and documentation can be obtained at:
% % http://www.ctan.org/pkg/stfloats
% % Do not use the stfloats baselinefloat ability as the IEEE does not allow
% % \baselineskip to stretch. Authors submitting work to the IEEE should note
% % that the IEEE rarely uses double column equations and that authors should try
% % to avoid such use. Do not be tempted to use the cuted.sty or midfloat.sty
% % packages (also by Sigitas Tolusis) as the IEEE does not format its papers in
% % such ways.
% % Do not attempt to use stfloats with fixltx2e as they are incompatible.
% % Instead, use Morten Hogholm'a dblfloatfix which combines the features
% % of both fixltx2e and stfloats:
% %
% % \usepackage{dblfloatfix}
% % The latest version can be found at:
% % http://www.ctan.org/pkg/dblfloatfix




% %\ifCLASSOPTIONcaptionsoff
% %  \usepackage[nomarkers]{endfloat}
% % \let\MYoriglatexcaption\caption
% % \renewcommand{\caption}[2][\relax]{\MYoriglatexcaption[#2]{#2}}
% %\fi
% % endfloat.sty was written by James Darrell McCauley, Jeff Goldberg and 
% % Axel Sommerfeldt. This package may be useful when used in conjunction with 
% % IEEEtran.cls'  captionsoff option. Some IEEE journals/societies require that
% % submissions have lists of figures/tables at the end of the paper and that
% % figures/tables without any captions are placed on a page by themselves at
% % the end of the document. If needed, the draftcls IEEEtran class option or
% % \CLASSINPUTbaselinestretch interface can be used to increase the line
% % spacing as well. Be sure and use the nomarkers option of endfloat to
% % prevent endfloat from "marking" where the figures would have been placed
% % in the text. The two hack lines of code above are a slight modification of
% % that suggested by in the endfloat docs (section 8.4.1) to ensure that
% % the full captions always appear in the list of figures/tables - even if
% % the user used the short optional argument of \caption[]{}.
% % IEEE papers do not typically make use of \caption[]'s optional argument,
% % so this should not be an issue. A similar trick can be used to disable
% % captions of packages such as subfig.sty that lack options to turn off
% % the subcaptions:
% % For subfig.sty:
% % \let\MYorigsubfloat\subfloat
% % \renewcommand{\subfloat}[2][\relax]{\MYorigsubfloat[]{#2}}
% % However, the above trick will not work if both optional arguments of
% % the \subfloat command are used. Furthermore, there needs to be a
% % description of each subfigure *somewhere* and endfloat does not add
% % subfigure captions to its list of figures. Thus, the best approach is to
% % avoid the use of subfigure captions (many IEEE journals avoid them anyway)
% % and instead reference/explain all the subfigures within the main caption.
% % The latest version of endfloat.sty and its documentation can obtained at:
% % http://www.ctan.org/pkg/endfloat
% %
% % The IEEEtran \ifCLASSOPTIONcaptionsoff conditional can also be used
% % later in the document, say, to conditionally put the References on a 
% % page by themselves.




% % *** PDF, URL AND HYPERLINK PACKAGES ***
% %
% %\usepackage{url}
% % url.sty was written by Donald Arseneau. It provides better support for
% % handling and breaking URLs. url.sty is already installed on most LaTeX
% % systems. The latest version and documentation can be obtained at:
% % http://www.ctan.org/pkg/url
% % Basically, \url{my_url_here}.




% % *** Do not adjust lengths that control margins, column widths, etc. ***
% % *** Do not use packages that alter fonts (such as pslatex).         ***
% % There should be no need to do such things with IEEEtran.cls V1.6 and later.
% % (Unless specifically asked to do so by the journal or conference you plan
% % to submit to, of course. )



% % HB extensions ==V====v20150211=============================================
% % HB Specific
% \newcommand{\hbAverage}[1]{\overline{#1}}

% % ======= HB ===============
% % HB helper
% \newcommand{\etal}{et al. }
% \newcommand{\hbie}{i.e.,}
% \newcommand{\hbeg}{e.g.,}


% \newcommand{\hbSection}[1]{\section{#1}}
% %\newcommand{\hbSection}[1]{}
% %

% \newcommand{\hbQuote}[1]{``\textsf{{\footnotesize #1}}''}
% %\newcommand{\hbQuote}[1]{{\small \textsf{#1}}}
% %\newcommand{\hbQuote}[1]{\textsf{#1}}
% %{\footnotesize aaa}


% %\newcommand{\hbIdea}[1]{}
% \usepackage{color}
% \newcommand{\hbIdea}[1]{{\color{red}{\scriptsize [{#1}]}}}

% % HB picture size
% %\newcommand{\myFigWidth}{.65\columnwidth}
% \newcommand{\myFigWidth}{8cm}

% % HB References
% \usepackage[hidelinks]{hyperref} %boxes hidden, remove hidelinks if boxes are desired.
% \newcommand{\reffig}[1]{Fig.~\ref{#1}}
% \newcommand{\refeq}[1]{Eq.~\ref{#1}}
% \newcommand{\reftbl}[1]{Table~\ref{#1}}
% \newcommand{\refsec}[1]{Sec.~\ref{#1}}
% \newcommand{\refthm}[1]{Theorem~\ref{#1}}
% \newcommand{\refthmA}[2]{\refthm{#1}(\ref{#2}}
% \newcommand{\reflem}[1]{Lemma~\ref{#1}}
% \newcommand{\refdef}[1]{Definition~\ref{#1}}
% \newcommand{\refexmp}[1]{Example~\ref{#1}}
% \newcommand{\refitem}[1]{(\ref{#1})}
% \newcommand{\refcite}[1]{ref~\cite{#1}}
% % ======= HB ===============

% %
% %\usepackage[yyyymmdd,hhmmss]{datetime}

% \usepackage[iso]{datetime}
% \newcommand{\hbNow}{--Draft-- v\today/\currenttime} % version


% %% HB Counter
% %\newcounter{cntrObservation}
% %\newcommand{\hCountDiff}{\stepcounter{cntrObservation}\textbf{Observation~\arabic{cntrObservation}.~}}
% %% HB Logic
% %\newcommand{\hDefinitionIFF}{\ensuremath{ \ \overset{\Delta}{\longleftrightarrow}} \ } % <-->
% %\newcommand{\hSuchThat}{\;|\;}
% %% HB enumerate
% %\usepackage{enumitem}
% %\newenvironment{hEnumerateRoman}
% %{\begin{enumerate}[label=\roman*.,leftmargin=*,topsep=-10pt,partopsep=0pt,parsep=0pt,itemsep=0pt]}
% %{\end{enumerate}}
% %\newenvironment{hEnumerateAlpha}
% %{\begin{enumerate}[label=\alph*) ,leftmargin=*,topsep=-10pt,partopsep=0pt,parsep=0pt,itemsep=0pt]}
% %{\end{enumerate}}
% % HB extensions ==A========================================================



% % correct bad hyphenation here
% \hyphenation{op-tical net-works semi-conduc-tor}


% \begin{document}
% %
% % paper title
% % Titles are generally capitalized except for words such as a, an, and, as,
% % at, but, by, for, in, nor, of, on, or, the, to and up, which are usually
% % not capitalized unless they are the first or last word of the title.
% % Linebreaks \\ can be used within to get better formatting as desired.
% % Do not put math or special symbols in the title.
% \title{
% %	Gender Compatibility in Selecting Partner
% 	Smart presentation of HTML pages
% % 	{\small \\ \hbNow}  % <<<<<<<<<<<<<<<<<<<<<<<<<<<<<<<<<<<<<<<<<<
% }
% %
% %
% % author names and IEEE memberships
% % note positions of commas and nonbreaking spaces ( ~ ) LaTeX will not break
% % a structure at a ~ so this keeps an author's name from being broken across
% % two lines.
% % use \thanks{} to gain access to the first footnote area
% % a separate \thanks must be used for each paragraph as LaTeX2e's \thanks
% % was not built to handle multiple paragraphs
% %

% \author{Kerim Can Macit% <-this % stops a space
% \thanks{
% H.~O.~Bingol  and Omer Basar are with 
% the Department of Computer Engineering, 
% Bogazici University, 
% Istanbul,
% 34342 Turkey 
% e-mail: bingol@boun.edu.tr.
% }% <-this % stops a space
% \thanks{Manuscript received December 1, 2015; revised XX XX, 2015.}}

% % note the % following the last \IEEEmembership and also \thanks - 
% % these prevent an unwanted space from occurring between the last author name
% % and the end of the author line. i.e., if you had this:
% % 
% % \author{....lastname \thanks{...} \thanks{...} }
% %                     ^------------^------------^----Do not want these spaces!
% %
% % a space would be appended to the last name and could cause every name on that
% % line to be shifted left slightly. This is one of those "LaTeX things". For
% % instance, "\textbf{A} \textbf{B}" will typeset as "A B" not "AB". To get
% % "AB" then you have to do: "\textbf{A}\textbf{B}"
% % \thanks is no different in this regard, so shield the last } of each \thanks
% % that ends a line with a % and do not let a space in before the next \thanks.
% % Spaces after \IEEEmembership other than the last one are OK (and needed) as
% % you are supposed to have spaces between the names. For what it is worth,
% % this is a minor point as most people would not even notice if the said evil
% % space somehow managed to creep in.



% % The paper headers
% % \markboth{Journal of \LaTeX\ Class Files,~Vol.~14, No.~8, August~2015}%
% % {Shell \MakeLowercase{\textit{et al.}}: Bare Demo of IEEEtran.cls for IEEE Communications Society Journals}
% % The only time the second header will appear is for the odd numbered pages
% % after the title page when using the twoside option.
% % 
% % *** Note that you probably will NOT want to include the author's ***
% % *** name in the headers of peer review papers.                   ***
% % You can use \ifCLASSOPTIONpeerreview for conditional compilation here if
% % you desire.




% % If you want to put a publisher's ID mark on the page you can do it like
% % this:
% %\IEEEpubid{0000--0000/00\$00.00~\copyright~2015 IEEE}
% % Remember, if you use this you must call \IEEEpubidadjcol in the second
% % column for its text to clear the IEEEpubid mark.



% % use for special paper notices
% %\IEEEspecialpapernotice{(Invited Paper)}




% % make the title area
% \maketitle

% % As a general rule, do not put math, special symbols or citations
% % in the abstract or keywords.
% \begin{abstract}
% 	ABSTRACT WILL BE HERE
% \end{abstract}

% % Note that keywords are not normally used for peerreview papers.
% \begin{IEEEkeywords}
% 	Mating,
% 	mate selection,
% 	mating preferences,
% 	parental investment,
% 	gender compatibility,
% 	evolution,
% 	online dating.
% \end{IEEEkeywords}






% % For peer review papers, you can put extra information on the cover
% % page as needed:
% % \ifCLASSOPTIONpeerreview
% % \begin{center} \bfseries EDICS Category: 3-BBND \end{center}
% % \fi
% %
% % For peerreview papers, this IEEEtran command inserts a page break and
% % creates the second title. It will be ignored for other modes.
% \IEEEpeerreviewmaketitle



% % 
% %\section{Introduction}
% % The very first letter is a 2 line initial drop letter followed
% % by the rest of the first word in caps.
% % 
% % form to use if the first word consists of a single letter:
% % \IEEEPARstart{A}{demo} file is ....
% % 
% % form to use if you need the single drop letter followed by
% % normal text (unknown if ever used by the IEEE):
% % \IEEEPARstart{A}{}demo file is ....
% % 
% % Some journals put the first two words in caps:
% % \IEEEPARstart{T}{his demo} file is ....
% % 
% % Here we have the typical use of a "T" for an initial drop letter
% % and "HIS" in caps to complete the first word.



% % ========================================================================
% \section{Introduction}

% \hbIdea{gender differences} % ====================
% \IEEEPARstart{T}{here} 
% %There 
% has been a long debate on how different male and female are~\cite{
% 	Darwin1871Book,
% 	Tannen1991Book,
% 	Buss2003Book,
% 	Gray2009Book,
% 	Liljeros2001Nature,
% 	Schmitt2008,
% 	CelaConde2009PNAS,
% 	Hyde2009PNAS,
% 	Szell2013SR,
% 	Blanch2015,
% 	Hyde2005,
% 	Zell2015}.
% \emph{Gender difference hypothesis} claims that 
% males and females are very different in their 
% personalities,
% abilities,
% interests,
% attitudes and 
% behavioral tendencies~\cite{
% 	Tannen1991Book,
% 	Buss2003Book,
% 	Gray2009Book,
% 	Liljeros2001Nature,
% 	Schmitt2008,
% 	CelaConde2009PNAS,
% 	Hyde2009PNAS,
% 	Szell2013SR,
% 	Blanch2015}.
% Some recent findings support this hypothesis.
% Men and women are different in many ways including
% sexual contacts~\cite{
% 	Liljeros2001Nature},
% in brain imaging studies~\cite{
% 	CelaConde2009PNAS},
% performance in Mathematics~\cite{
% 	Hyde2009PNAS}, 
% or chess~\cite{
% 	Blanch2015},
% even online games~\cite{
% 	Szell2013SR}.
% On the other hand, 
% some investigations support the \emph{gender similarity hypothesis} 
% which claims that 
% the difference is not as big as one expects~\cite{
% 	Hyde2005,
% 	Zell2015}.

% \hbIdea{hunter-gatherer theory} % ====================
% One possible evolutionary explanation of the gender difference is 
% the \emph{hunter-gatherer theory of spatial sex differences} 
% proposed by Silverman and Earl~\cite{
% 	Silverman1992}. 
% It claims that there has been division of labor between men and women
% as early as the time of hunter-gatherers.
% Males are primarily hunters and females primarily foraged.
% This affects the cognitive development since
% \hbQuote{
% 	tracking and killing animals entail different kinds of spatial problems 
% 	than does foraging for edible plants; 
% 	thus,
% 	adaptation would have favored diverse spatial skills between sexes 
% 	throughout much of their evolutionary history
% }.
% Hence, 
% it calls for different spatial skills such as
% mental rotations,
% map reading,
% maze learning 
% for male.
% and
% ability to learn, recognize, remember spatial configurations of objects,
% and their spatial relationships 
% for female.




% % ========================================================================
% \subsection{Mating and Parental Investment}

% \hbIdea{mating and parental investment theory} % ====================
% Mating is important for evolution.
% %There are gender differences in mating, too. 
% In many species, it has been observed that 
% male and female have different strategies in mate 
% selection.

% We need an evolutionary theory 
% to explain differences in mating strategies. 
% One such theory is 
% Trivers' 
% \emph{parental investment theory} 
% which is based on parental investment~\cite{
% 	Trivers1972}.
% %An evolutionary theory 
% %that explains differences in mating strategies 
% %is \emph{parental investment theory} proposed by 
% %Trivers~\cite{
% %	Trivers1972}.
% Offsprings, having support from parents, have better chance to survive, 
% hence to reproduce.
% Therefore, 
% evolution calls for parental support.
% He carefully defines \emph{parental investment} as
% \hbQuote{
% 	any investment by the parent in an individual offspring 
% 	that increases the offspring's chance of surviving 
% 	(and hence reproductive success)
% 	at the cost of the parent's ability to invest in other offspring
% }.

% \hbIdea{mating strategies} % ====================
% Parental investment is quite uneven between male and female
% in many species~\cite{
% 	Trivers1972}.
% Therefore both genders evolutionarily developed mating strategies 
% which are clearly different~\cite{
% 	Buss1989,
% 	Buss1993,
% 	Buss2003Book,
% 	Buss2006}.
% \hbIdea{female strategy} % ====================
% In female mammal, 
% hence human female, 
% is forced to select quality,
% since she cannot choose quantity.
% Human female makes mandatory high investment in offspring compared to male,
% if one considers nine months of gestation, childbirth, lactation, nurturing.
% Therefore, 
% she looks for supporting male in her mate selection.
% She prefers a male who 
% not only have the resources to support her 
% but also willing to commit these resources to her.
% That explains female preference for \emph{long-term} commitment.
% \hbIdea{female strategy} % ====================
% On the other hand, 
% human male can choose quantity.
% He is reluctant to long-term commitment. 
% He has a tendency for \emph{short-term} relations
% which increase his chances to reproduce offsprings.
% This quality versus quantity paradigm is clearly a conflict 
% that has to be resolved.
% Female, 
% who invest more in offspring,
% should be more choosy selecting mate
% (\emph{intersexual attractio}n)
% and
% male,
% who invest less,
% should compete to access the opposite sex
% (\emph{intrasexual competition})~\cite{
% 	Trivers1972}.





% % ========================================================================
% \subsection{Properties in Mate Selection}

% %How do you judge fertility or reproductive value?
% %The only clue is the observations one can make.
% %For example, 
% %by evolution,
% %male has learned to look for 
% %(i)~physical features (\hbeg
% %	symmetry,
% %	good muscle tone,
% %	absence of lesions),
% %(ii)~observable behavior (\hbeg
% %	sprightly,
% %	high activity level),
% %(iii)~social reputation (\hbeg
% %	knowledge gathered from others)~\cite{
% %	Buss1989,
% %	Buss1993,
% %	Buss2006}.
% %
% %By evolution, 
% %Male looks for 
% %full lips,
% %smooth skin,
% %lustrous hair,
% %low ratio of hips to waist
% %for fertility~\cite{
% %	Buss1989,
% %	Buss2006}.
% %
% %Female prefers male having properties such as 
% %wealth, 
% %masculine physical atributes
% %such as 
% %height,  


% \hbIdea{mating properties} % ====================
% Properties, 
% that increase the chance of mating, 
% become crucial in this respect~\cite{
% 	Buss1989,
% 	Buss1993,
% 	Buss2006,
% 	Trivers1972,
% 	Kenrick1992, % Male-older
% 	Gillis1980, % Male-taller
% 	Pawlowski2000Nature,  % Male-taller
% 	Swami2008,  % Male-taller
% 	Silverman1992,  % Male-taller
% 	Buss2003Book}.  % Male-taller
% \hbIdea{age} % ====================
% In terms of evolution,
% % 
% (a)~\emph{fertility},
% \hbie~immidiate probability of conception,
% and
% %
% (b)~\emph{reproductive value}
% \hbie~future reproductive potential,
% are the top two properties for both 
% gender~\cite{
% 	Buss2006}.
% 	They are age related.
% Clearly, health is also very important.
% %the next property one looks for.
% So young healthy mate should be the choice in all species.
% Human male just does that.
% But human female has other issues therefore being young is not enough.
% She is looking for male that will provide parental support for her offspring,
% that is,
% he should 
% %
% (a)~have the resources and 
% %
% (b)~willingness to commit these resources to her offspring.
% In order to collect resources he needs time.
% Therefore he may not be that young after all.
% \hbIdea{physical} % ====================
% While female is busy providing parental investment to her offspring,
% she expects male 
% to provide food and shelter.
% He is also expected to protect her and her offspring.
% So physically masculine male should be preferred. 
% \hbIdea{expect} % ====================
% So we expect that younger female and 
% ``superior'' male partners.

% Empirical evidences support these 
% deductions~\cite{
% 	Schmitt2008,
% 	Buss1989,
% 	Buss2003Book,
% 	Kenrick1992, % Male-older
% 	Gillis1980, % Male-taller
% 	Pawlowski2000Nature,  % Male-taller
% 	Swami2008,  % Male-taller
% 	Silverman1992}.  % Male-taller
% %Human male and female have different preferences in mate 
% %selection~\cite{
% %	Kenrick1992, % Male-older
% %	Gillis1980, % Male-taller
% %	Pawlowski2000Nature,  % Male-taller
% %	Swami2008,  % Male-taller
% %	Silverman1992,  % Male-taller
% %	Buss2003Book}.  % Male-taller
% Investigations on partners has revealed that
% \hbIdea{age} % ====================
% male is older~\cite{
% 	Kenrick1992},
% \hbIdea{physical} % ====================
% taller~\cite{
% 	Gillis1980,
% 	Pawlowski2000Nature,
% 	Swami2008}
% than female.
% This is a universal pattern across cultures~\cite{
% 	Buss1989,
% 	Schmitt2008}.




% %% =========================================================================
% %\subsection{Online Dating}
	


% %Data
% %that have never been reachable 
% %before is available on internet~\cite{
% %	Szell2010PNAS,
% %	Centola2010Science,
% %	Rocha2010PNAS,
% %	Szell2013SR,
% %	Kramer2014PNAS}.
% %\hbIdea{ODS} % ====================
% %Online dating cannot be kept away from this 
% %trend~\cite{
% %	Barraket2008JSociology,
% %	Bingol2012PartnerArxiv,
% %	Norcie2013LNCS,
% %	Zhao2014IS}.




% %% ========================================================================
% %\subsection{Compatibility of Preferences}

% \hbIdea{Compatibility of Preferences} % ====================
% Since mating requires agreement of both parties, 
% although men and women have different preferences in mate selection, 
% there should be compatibility in these preferences. 
% We can ask the following question.
% How compatible are the preferences of two genders in a given property?

% We will use the data obtained from an online dating site.
% First we carefully define mating in our data set.
% Then we aggregate the properties of partners that an individual selects as mate.
% Finally we search for patterns in the properties for mating behavior.

% %
% %One such cross cultural study is done by 
% %Buss~\cite{
% %	Buss1989}.
% %They investigate human mate preferences in 37 cultures with  
% %4,601 male, 5,446 female, in total 10,047, 
% %participants. 
% %Participants are asked a series of questions about their preferences in selection of a mate.
% %One of the questions is 
% %the age difference they prefer to have.
% %On the average,
% %males prefer 2.66 years younger, 
% %females prefer 3.42 years older mate than themselves.
% %Note that this is just a desire which may not be realized
% %due to many factors including local availability of mates, sex ratio.
% %In the same study, based on another source,
% %it is estimated that the average actual age difference of married couples is 2.99 years 
% %which is close to the average of 2.66 and 3.42.
% %Participants are also asked to rate given a set of characteristics 
% %in the order of importance in choosing a mate.
% %%The rating scale ranges from 
% %%3 (indispensable) to 
% %%0 (irrelevant or unimportant).
% %Among those characteristics, 
% %``good financial prospects'' is related to our study.
% %They observed that, 
% %except Spain, 
% %in all 36 countries females value 
% %``good financial prospects'' significantly more than males.




% % =========================================================================
% \section{Method}




% % =========================================================================
% \subsection{Data Set}
	

% \hbIdea{online dating data} % ====================
% We investigate the data of a large Turkish online dating site 
% for compatibility of mating preferences~\cite{
% 	Bingol2012PartnerArxiv}.
% There are 4,500,000 registered users in total.
% More than 3,000 new users register daily. 
% A user stay in the system for 3 months, on the average. 
% Many of them come back, later;
% sometimes as a new user.
% The daily activity is also quite large such as 
% 50,000 user logins,
% 500,000 massage transactions, 
% 5,000 photo uploads, and 
% 20,000 votes.




% % =========================================================================
% \subsection{Definition of Mating}

% \hbIdea{online dating} % ====================
% A typical online dating system enables its user to 
% find partner that best matches one's desires.
% Each user defines his user profile.
% An initiator, predominantly male, 
% selects a potential partner by examining her profile and 
% sends her a message.
% If there is a positive response from the receiver, 
% then more messages are exchanged
% which hopefully leads to a face-to-face meeting.

% \hbIdea{partner criteria} % ====================
% When do we say that male and female are mating?
% Online dating site has lots of information about the virtual world, 
% but there is usually no information whether 
% the male and the female really make a mating partner in the physical world.
% Any action in an online dating site is clearly an attempt for mating
% but is it sufficient to be considered as mating?
% For example,
% %(i)~
% just sending a message,
% %(ii)~
% getting a message in response, 
% or 
% %(iii)~
% some more exchange of messages 
% should not be enough in the world of online dating which is full of these.
% %make them a partner.

% %Usually online dating sites do not have any information 
% %whether 
% %the male and the female really make a partner in the physical world.
% Therefore we select  
% the most restricted criteria of mutual interest, 
% that is available in our data set,
% which is virtual gifts~\cite{Bingol2012PartnerArxiv}.
% Receiving a \emph{virtual gift}, 
% which is usually a picture of a flower, 
% is considered  a ``value'' in this virtual society.
% We have even observed that some user sent virtual gifts to themselves.
% This value is probably due to
% (i)~the virtual gifts one receives is visible to all,
% (ii)~they are not free, 
% \hbie\ one has to purchase virtual gifts in order to sent, and
% (iii)~only qualified users can sent virtual gifts.
% Since unpaid male members are not qualified to sent gifts, 
% able to sent gifts may be considered as an indication of wealth.

% There are $276,210$ male and 
% $483,963$ female users that are qualified to send virtual gift in the system. 
% Among those, only 
% $29,274$ male and 
% $14,981$ female,
% in total $N = 44,253$, 
% users reciprocally exchange virtual gifts.
% Hence we define a pair as (mating) \emph{partners} if
% they not only exchange messages, 
% but also 
% send at least one gift to, 
% and 
% receive at least one gift from each other.

% Note that  
% this definition is based on actual behaviors of users in the online dating site.
% We have the ``actual'' partners,
% that is,
% they mutually agree to ``mate'' 
% as far as we can trace in our online dating site, 
% rather than 
% a ``theoretical'' partner one wishes to have 
% as he answers the questions of a survey
% as in the case of \refcite{
% 	Buss1989}.
% We have such theoretical data in user profiles, too. 
% Users specify what properties, 
% such as the age, height, 
% they look for in their potential partners.
% This data is noisy.
% Users are not consistent.
% They claim something and does something else.
% For example,
% someone claims that he prefers women taller than 170~cm but
% does not hesitate to be partner with a 160~cm.
% Such behavior is clearly difficult to register in questionaries.
% In this respect our actual data deserves special attention.

	




% % =========================================================================
% \subsection{Properties of the Mate}

% \hbIdea{properties of the mate} % ====================
% Once we have identified the partners,
% we investigate the properties of the mate.
% As expected, 
% user $i$ becomes partner with many others as time goes.
% Each partner of $i$ may have different value for property $p$.
% The average of the properties of the partners of $i$ 
% is given as
% \[
% 	\hbAverage{p_{i}} = \frac{1}{|C_{i}|}  \sum_{j \in C_{i}} p_{j}
% \]
% where 
% $p_{j}$ denotes the property $p$ as it is defined in user $j$'s profile
% and 
% $C_{i}$ is the set of users that $i$ partnered with.
% We interpret this as
% user $i$ has a tendency to select partners having value of 
% $\hbAverage{p_{i}}$ in property $p$ .
% %That is,
% %$i$ \emph{prefers} $\hbAverage{p_{i}}$ in her partners.
% Hence, 
% we call $\hbAverage{p_{i}}$ as the \emph{preferred value} for $i$.
% %
% Instead of using the preferred value directly,
% we compare one's own value
% to the preferred value that one looks in his partners.
% The \emph{preferred difference} of $i$, 
% in property $p$, 
% is defined as
% %\begin{linenomath*}
% \[
% 	\Delta p_{i} = p_{i} - \hbAverage{p_{i}}.
% \]
% %\end{linenomath*}
% Note that $\Delta p_{i}$ can be negative or positive. 
% If $\Delta p_{i}$ is around $0$
% then the user prefers partners with similar properties with him, 
% i.e. homophily~\cite{
% 	McPherson2001,
% 	Centola2011Science}.
% For example, 
% if the property is height
% and
% if user $i$ has a tendency for taller partner in her selection,
% then $\Delta p_{i}$ would be negative.




% % =========================================================================
% \subsection{Distribution of Preferred Differences}

% \hbIdea{property distribution} % ====================
% We can extend these concepts from individual $i$ to a group of people.
% Then, frequency of people with the same preferred difference 
% makes a probability distribution,
% which we call \emph{preferred difference distribution}.
% Having 
% all women as one group, and 
% all men as another group,
% we obtain two preferred difference distributions
% $f(x)$ and $m(x)$ of females and males, respectively. 




% % =========================================================================
% \section{Results}

% \hbIdea{findings} % ====================
% The statistical parameters of the preferred differences in 
% age, height, education and income are given in
% \reftbl{tbl:statistics}.
% Columns 
% $\mu_{m}$,  $\mu_{f}$, 
% and
% $\sigma_{m}$, $\sigma_{f}$ are the
% averages and standard deviations of males and females, respectively.
% The distributions of the preferred differences
% %for 
% %height, age, education and income 
% are given in  \reffig{fig:propertyDifference}.
% We first focus on the averages,
% and leave the discussion of 
% distributions and 
% their compatibility, 
% the last column of \reftbl{tbl:statistics}, 
% later.




% % =========================================================================
% \subsection{Average of Preferred Differences}

% In all four properties in \reftbl{tbl:statistics},
% there is a distinct pattern.
% The averaged preferred differences 
% for males, $\mu_{m}$, are all negative 
% and
% that of females are all positive. 
% This observation indicates that in all four properties, 
% whatever the metric is used to measure the property,
% males prefer ``inferior'' females
% and females prefer ``superior'' males compared to themselves.


% \hbIdea{age} % ====================
% \textbf{Age.}
% According to evolutionary theories,
% we expect to see younger female and older male in partners.
% Our findings confirms that.
% We observe that,
% on the average, 
% males mate with females $2.90$ years younger than themselves,
% and
% females mate with males $2.74$ years older.
% Our findings are in agreement with 
% males prefer 2.66 years younger, 
% females prefer 3.42 years older mate than themselves reported by Buss~\cite{
% 	Buss1989}.
% %
% %In a comparable cross cultural study ($N = 10,047$),
% %Buss reports that, 
% %on the average,
% %males prefer 2.66 years younger, 
% %females prefer 3.42 years older mate than themselves~\cite{
% %	Buss1989}.
% %Our findings in agreement with this.



% %They investigate human mate preferences in 37 cultures with  
% %4,601 male, 5,446 female, in total $N = 10,047$, 
% %participants. 
% %Participants are asked a series of questions about their preferences in selection of a mate.
% %It is reported that,

% %
% %One such cross cultural study is done by 
% %Buss~\cite{
% %	Buss1989}.
% %They investigate human mate preferences in 37 cultures with  
% %4,601 male, 5,446 female, in total 10,047, 
% %participants. 
% %Participants are asked a series of questions about their preferences in selection of a mate.
% %One of the questions is 
% %the age difference they prefer to have.
% %On the average,
% %males prefer 2.66 years younger, 
% %females prefer 3.42 years older mate than themselves.
% %Note that this is just a desire which may not be realized
% %due to many factors including local availability of mates, sex ratio.
% %In the same study, based on another source,
% %it is estimated that the average actual age difference of married couples is 2.99 years 
% %which is close to the average of 2.66 and 3.42.

% %	Participants are also asked to rate given a set of characteristics 
% %	in the order of importance in choosing a mate.
% %	%The rating scale ranges from 
% %	%3 (indispensable) to 
% %	%0 (irrelevant or unimportant).
% %	Among those characteristics, 
% %	``good financial prospects'' is related to our study.
% %	They observed that, 
% %	except Spain, 
% %	in all 36 countries females value 
% %	``good financial prospects'' significantly more than males.
% %	====

% %The average preferred difference in age is 
% %$-2.90$ and $2.74$ years for males and females, respectively.
% %Considering the extreme values 
% %$-7.38$ (Zambian males ) and $5.10$ (Iranian females) years,
% %the averages
% %$-2.66$ for male and $3.42$ for female  
% %of \refcite{Buss1989}
% %are in agreement with our figures.
% %One observation is that male and female averages close to different countries.
% %For males, 
% %the countries close to $-2.90$ are 
% %	Poland ($-2.85$),
% %	Israel (Jewish) ($-2.88$),
% %	Brazil ($-2.94$), and
% %	Venezuela ($-2.99$).
% %For females,
% %close to $2.74$ are
% %	Canada (English) ($2.72$),
% %	Netherlands ($2.72$),
% %	Ireland ($2.78$), and
% %	Finland ($2.83$).
% %
% %The statistical parameters 
% %$\mu_{m}$,  $\mu_{f}$, 
% %and
% %$\sigma_{m}$, $\sigma_{f}$ of preferred difference distributions of genders
% %are given in \reftbl{tbl:statistics}.








% \hbIdea{difference in partner selection} % ====================
% \textbf{Height.}
% Our findings on preferred difference in height, given in \reffig{fig:propertyDifferenceHeight}, agree with the previous work.
% People usually interact with people who have similar 
% characteristics~\cite{
% 	McPherson2001,
% 	Centola2011Science}.
% For example 
% no drastic height differences between partners are observed.
% That is, 
% tall male partners with tall female, and short with 
% short~\cite{
% 	Gillis1980,
% 	Swami2008}.
% Although there are males prefer females 
% $30$\,cm shorter
% or 
% $10$\,cm taller 
% in \reffig{fig:propertyDifferenceHeight},
% they are rare.
% %This is a manifestation of homophily in height.
% %tall men partners with tall women.
% Majority are accumulated around the average
% which is a
% manifestation of homophily in height.

% Yet, 
% there are distinct differences between the preferences of male and female
% %This is not exactly the case for partner selection.
% %Different genders have different preferences in properties 
% when it comes to partner selection,
% such as height~\cite{
% 	Gillis1980,
% 	Swami2008,
% 	Pawlowski2000Nature}.
% Male is usually taller then female in partners,
% which is called \emph{male-taller norm}~\cite{
% 	Gillis1980}.
% The averages in \reftbl{tbl:statistics}
% agree with the male-taller norm.
% On the average, 
% male prefers female 
% $11.12$\,cm shorter.
% Similarly, 
% on the average
% females prefer male 
% $11.37$\,cm taller.

% \textbf{Income and Education.}
% We observe similar pattern in income and education, too,
% namely,
% males prefer negative and females prefer positive differences.
% Here the numbers cannot be compared with other works directly
% since 
% users are asked to select one bin out of many bins 
% which are organized in a consistent but an arbitrary way. 
% They are consistent in the sense that the larger the bin number,
% the more educated or higher income.
% The bins in education are related to the number of schooling years
% such as
% graduate of primary school, or
% of college.
% The bins in income field represent monthly income such as 
% bin 2: $500 < x < 1000$,
% bin 3: $1000 < x < 2000$.

% %Both income and education are selected from
% %For example,
% %for the income field of data, 
% %user is allow to select one out of five income bins, namely, 
% %bin 1: $x < 500$, 
% %bin 2: $500 < x < 1000$,
% %bin 3: $1000 < x < 2000$,
% %bin 4: $2000 < x < 3000$, and 
% %bin 5: $3000 < x$. 
% %So what we can do is to compare the bins of the partners.


% % =========================================================================
% % TABLE HERE 
% % {tbl:statistics}
% % -------------------------------------------------------------------------
% \begin{table}%[htdp]
% \caption{Comparison of male and female distributions}
% \begin{center}
% %\begin{tabular*}{\hsize}{@{\extracolsep{\fill}}|l|rr|rr|c|}
% \begin{tabular*}{\hsize}{@{\extracolsep{\fill}}lrrrrc}
% % -------------------------------
% \hline
% Property
% &\multicolumn{2}{c}{Averages}
% &\multicolumn{2}{c}{Standard Deviations}
% &Compatibility\\
% % 
% &$\mu_{m}$
% &$\mu_{f}$
% &$\sigma_{m}$
% &$\sigma_{f}$
% &$\rho$\\
% %
% \hline
% %
% Height (cm)		
% &-11.12 
% &11.37 
% &6.76
% &7.09
% &0.90\\
% %
% Education (bin)
% &-0.36
% &0.34
% &1.35
% &1.40
% &0.92\\
% %
% Age (year)			
% &-2.90
% &2.74
% &5.06
% &5.23
% &0.94\\
% %
% Income (bin)
% &-0.93
% &0.99
% &1.28
% &1.32
% &0.95\\
% \hline
% % -------------------------------
% \end{tabular*}
% \end{center}
% \label{tbl:statistics}
% \end{table}
% % -------------------------------------------------------------------------
% % =========================================================================



% % =========================================================================
% % FIG HERE 
% % {fig:propertyDifference}
% % ------------------------------------------------------------------------
% %\begin{figure}[ht]
% \begin{figure*}%[ht]
% \centering
% %\begin{center}
% 	%
% 	\subfloat[Height]{
% %		\includegraphics[width=.65\columnwidth]{imgDiff_Height}
% 		\includegraphics[width=\myFigWidth]{imgDiff_Height}
% 		\label{fig:propertyDifferenceHeight}
% 	}
% 	%
% 	\hfil
% 	\subfloat[Age]{
% 		\includegraphics[width=\myFigWidth]{imgDiff_Age}
% 		\label{fig:propertyDifferenceAge}
% 	}\\
% 	%
% %	\hfil
% 	\subfloat[Education]{
% 		\includegraphics[width=\myFigWidth]{imgDiff_Education}
% 		\label{fig:propertyDifferenceEducation}
% 	}
% 	%
% 	\hfil
% 	\subfloat[Income]{
% 		\includegraphics[width=\myFigWidth]{imgDiff_Income}
% 		\label{fig:propertyDifferenceSalary}
% 	}
% %	\subfloat[BMI]{
% %		\includegraphics[scale=0.34]{imgDiff_BMI}
% %		\label{fig:propertyDifferenceBMI}
% %	}
% %	\subfloat[Body Type]{
% %		\includegraphics[scale=0.34]{imgDiff_BodyType}
% %		\label{fig:propertyDifferenceBodyType}
% %	} 
% 	\caption{
% 		Preferred difference distributions in 
% 		height, 
% 		age, 
% 		education, and
% 		income.
% 	}
% 	\label{fig:propertyDifference}
% %\end{center}
% \end{figure*}
% %\end{figure}
% % ------------------------------------------------------------------------
% % =========================================================================




% % =========================================================================
% \subsection{Distributions of Preferred Differences}

% Note that the average preferred differences of males and females are very close to each other in \reftbl{tbl:statistics}.
% The standard deviations are also very close to each other.
% If we assume that the male and female distributions are gaussian distributions,
% the distributions should be very similar
% as if female distribution is obtained by shifting the male distribution.
% This cannot be a coincidence and deserves further study.
% It seems males and females complement each other in the preferred differences.
% Since we have not only the averages and standard deviations, 
% but also the distributions,
% we can further investigate the distributions for compatibility.

% Distributions of preferred differences 
% in age, education, and salary are given in 
% \reffig{fig:propertyDifferenceAge},
% \reffig{fig:propertyDifferenceEducation}, and
% \reffig{fig:propertyDifferenceSalary}, respectively.
% %Similar findings are observed 
% %Besides homophily, 
% %one observes that the average preferred differences of opposite genders 
% %are in perfect harmony as seen in \reftbl{tbl:statistics}.
% %Additional properties, namely body mass index and body type, given in \cite{Bingol2015si} are also in agreement. 
% As we have deducted from \reftbl{tbl:statistics},
% for all properties in \reffig{fig:propertyDifference},
% the bell-shaped curves of male and female are also resemble to each other. 
% One notices that
% male curves are left-shifted, 
% and female curves are right-shifted
% with respect to the $y$-axis. 
% %%This is also observed in \reftbl{tbl:statistics} 
% %%as all male averages are negative and
% %%female ones are positive.


% %We interpret this, 
% %as males prefer female with ``lower'' qualification. 
% %Similarly females prefer male with ``higher'' qualifications.
% %That is, 
% %compared to themselves, 
% %females prefer 
% %taller and older males with 
% %better education and higher income.

% \hbIdea{compatible} % ====================
% In order to get better understanding,
% consider a simplified example given in \reffig{fig:propertyDifferenceDummy}.
% % *****************
% Note that
% females that prefer $\Delta p = x$
% matches with 
% males that prefers $\Delta p = -x$.
% Therefore,
% we should not compare the distribution $f(x)$ of females
% not with $m(x)$ of males as we previously thought.
% We should compare $f(x)$ with $m(-x)$,
% the symmetric graph with respect to the $y$-axis.
% We make a reasonable assumption that 
% there are equal number of men and women.
% Then
% $\min\{ {f(x), m(-x)} \}$ of the women who prefer $\Delta p = x$ are matched.
% Then, 
% the \emph{compatibility} of two distributions can be measured by means of the ratio of matched women given as
% \[
% 	\rho = \sum_{x} \min\{ {f(x), m(-x)} \}
% \]
% where summation is taken over all possible values of $x$.
% This is a well-defined metric since 
% the ratio of matched women is equal to that of men.

% In \reftbl{tbl:statistics}, 
% the properties are listed in ascending order in compatibility.
% Height is the property with the lowest compatibility.
% Even for this case,
% $90~\%$ of the population can find a satisfying partner.
% Interestingly,
% income has the highest compatibility and age comes next. 


% % =========================================================================
% % FIG HERE 
% % {fig:propertyDifferenceDummy}
% % ------------------------------------------------------------------------
% \begin{figure}%[ht]
% \centering
% %\begin{center}
% %	\includegraphics[width=.8\columnwidth]{imgDiff_Dummy}
% 	\includegraphics[width=\myFigWidth]{imgDiff_Dummy}
% 	\caption{
% 		Distribution of difference in 
% 		a dummy property $p$.
% 		We assume that male female populations are the same.
% 		%
% 		(i)~$50\%$ of women and 
% 		$60\%$ of men prefer no differences in $p$.
% 		Hence $50\%$ matches for $\Delta p = 0$.
% 		%
% 		(ii)~$20\%$ of women who prefer difference of $\Delta p = 1$ 
% 		are to match with
% 		$10\%$ of men who prefer differences of $\Delta p = -1$.
% 		Only $10\%$ matches for $\Delta p = 1$.
% 		%
% 		(iii)~$30\%$ of women who prefer difference of $\Delta p = -1$ 
% 		are exactly match with
% 		$30\%$ of men who prefer differences of $\Delta p = 1$.
% 		That is, $30\%$ matches for $\Delta p = 1$.
% 		%
% 		In total $90\%$ of women are match.
% 		Hence male female compatibility is $\rho = 0.90$.
% 	}
% 	\label{fig:propertyDifferenceDummy}
% %\end{center}
% \end{figure}
% % ------------------------------------------------------------------------
% % =========================================================================



% % =========================================================================
% \hbSection{Discussion}

% \hbIdea{discussion} % ====================
% In real life, 
% men and women behave differently.
% Our findings show that 
% the virtual world of online dating is another manifestation of this difference.
% %
% (i)~While male prefers women with lower qualifications in every property 
% that we have investigated, 
% women just do the opposite. 
% %
% (ii)~Interestingly, the preferences of men and women match to each other so that 
% the number of dissatisfied is minimized.
% Due to lack of space, we do not report here but 
% we have also observed similar findings in body mass index and body type, 
% too~\cite{
% 	Bingol2012PartnerArxiv}. 
	
% We can explain this evolutionary dynamic.
% Suppose we start with individual with uniform one individual prefers difference far away from the
% We are not in that position but 
% it would be nice if we could provide explanations for these findings.
% It may be possible to find some evolutionary 
% explanation~\cite{
% 	Kenrick1992,
% 	Silverman1992,
% 	Pawlowski2000Nature}.

% %We want to point out two more observations. 
% %%
% %(i)~The data clearly indicates that men take the first move in partner selection. 
% %Then, women have the right to accept or reject. 
% %%
% %(ii)~On the average females make more partners than males,
% %since there are 
% %$29,274$ male and 
% %$14,981$ female users to make partners,
% %there is 1 female for 2 males.
% %Note that this disagrees with 
% %men having more sexual partners than women~\cite{
% %	Liljeros2001Nature}.
% %It might be the case that 
% %this ratio changes 
% %when it comes to physical relationship.



% \hbIdea{warnings} % ====================
% One needs to be careful on a number of issues in a study like that.
% % 
% (i)~One has to keep in mind that 
% the findings could be culture dependent.
% %
% (ii)~The profile is based on user's claim, 
% that is, 
% it may be misleading. 
% On the other hand, 
% stretching the properties too far would not be a good strategy 
% since unfaithful declaration, 
% such as declared as slim while being obese, 
% would be an obstacle to further the relationship
% when the time comes to meet face-to-face~\cite{
% 	Norcie2013LNCS,
% 	Ellison2013}.
% So we assume that users are closed to what they claim to be.
% %
% (iii)~Privacy is the most important issue for such an investigation.
% In this study no data left the company. 
% All the data processing is done at their site.
% No individual personal information is used.
% Only statistical data such as given in \reffig{fig:propertyDifference} is shared with us.

% \section*{Acknowledgment}
% This work was partially supported 
% by Bogazici University Research Fund, BAP-2008-08A105, 
% by the Turkish State Planning Organization (DPT) TAM Project, 2007K120610,
% and
% by COST action MP0801.








% % An example of a floating figure using the graphicx package.
% % Note that \label must occur AFTER (or within) \caption.
% % For figures, \caption should occur after the \includegraphics.
% % Note that IEEEtran v1.7 and later has special internal code that
% % is designed to preserve the operation of \label within \caption
% % even when the captionsoff option is in effect. However, because
% % of issues like this, it may be the safest practice to put all your
% % \label just after \caption rather than within \caption{}.
% %
% % Reminder: the "draftcls" or "draftclsnofoot", not "draft", class
% % option should be used if it is desired that the figures are to be
% % displayed while in draft mode.
% %
% %\begin{figure}[!t]
% %\centering
% %\includegraphics[width=2.5in]{myfigure}
% % where an .eps filename suffix will be assumed under latex, 
% % and a .pdf suffix will be assumed for pdflatex; or what has been declared
% % via \DeclareGraphicsExtensions.
% %\caption{Simulation results for the network.}
% %\label{fig_sim}
% %\end{figure}

% % Note that the IEEE typically puts floats only at the top, even when this
% % results in a large percentage of a column being occupied by floats.


% % An example of a double column floating figure using two subfigures.
% % (The subfig.sty package must be loaded for this to work.)
% % The subfigure \label commands are set within each subfloat command,
% % and the \label for the overall figure must come after \caption.
% % \hfil is used as a separator to get equal spacing.
% % Watch out that the combined width of all the subfigures on a 
% % line do not exceed the text width or a line break will occur.
% %
% %\begin{figure*}[!t]
% %\centering
% %\subfloat[Case I]{\includegraphics[width=2.5in]{box}%
% %\label{fig_first_case}}
% %\hfil
% %\subfloat[Case II]{\includegraphics[width=2.5in]{box}%
% %\label{fig_second_case}}
% %\caption{Simulation results for the network.}
% %\label{fig_sim}
% %\end{figure*}
% %
% % Note that often IEEE papers with subfigures do not employ subfigure
% % captions (using the optional argument to \subfloat[]), but instead will
% % reference/describe all of them (a), (b), etc., within the main caption.
% % Be aware that for subfig.sty to generate the (a), (b), etc., subfigure
% % labels, the optional argument to \subfloat must be present. If a
% % subcaption is not desired, just leave its contents blank,
% % e.g., \subfloat[].


% % An example of a floating table. Note that, for IEEE style tables, the
% % \caption command should come BEFORE the table and, given that table
% % captions serve much like titles, are usually capitalized except for words
% % such as a, an, and, as, at, but, by, for, in, nor, of, on, or, the, to
% % and up, which are usually not capitalized unless they are the first or
% % last word of the caption. Table text will default to \footnotesize as
% % the IEEE normally uses this smaller font for tables.
% % The \label must come after \caption as always.
% %
% %\begin{table}[!t]
% %% increase table row spacing, adjust to taste
% %\renewcommand{\arraystretch}{1.3}
% % if using array.sty, it might be a good idea to tweak the value of
% % \extrarowheight as needed to properly center the text within the cells
% %\caption{An Example of a Table}
% %\label{table_example}
% %\centering
% %% Some packages, such as MDW tools, offer better commands for making tables
% %% than the plain LaTeX2e tabular which is used here.
% %\begin{tabular}{|c||c|}
% %\hline
% %One & Two\\
% %\hline
% %Three & Four\\
% %\hline
% %\end{tabular}
% %\end{table}


% % Note that the IEEE does not put floats in the very first column
% % - or typically anywhere on the first page for that matter. Also,
% % in-text middle ("here") positioning is typically not used, but it
% % is allowed and encouraged for Computer Society conferences (but
% % not Computer Society journals). Most IEEE journals/conferences use
% % top floats exclusively. 
% % Note that, LaTeX2e, unlike IEEE journals/conferences, places
% % footnotes above bottom floats. This can be corrected via the
% % \fnbelowfloat command of the stfloats package.




% \section{Conclusion}
% The conclusion goes here.





% % if have a single appendix:
% %\appendix[Proof of the Zonklar Equations]
% % or
% %\appendix  % for no appendix heading
% % do not use \section anymore after \appendix, only \section*
% % is possibly needed

% % use appendices with more than one appendix
% % then use \section to start each appendix
% % you must declare a \section before using any
% % \subsection or using \label (\appendices by itself
% % starts a section numbered zero.)
% %


% \appendices
% \section{Proof of the First Zonklar Equation}
% Appendix one text goes here.

% % you can choose not to have a title for an appendix
% % if you want by leaving the argument blank
% \section{???}

% \hbIdea{large data} % ====================
% Investigation of mating preferences
% on larger population in traditional ways,
% such as one-to-one interviews, 
% becomes prohibitively difficult.
% Thanks to Internet,
% very interesting data sets~\cite{
% 	Centola2010Science, 
% 	Rocha2010PNAS,
% 	Szell2010PNAS,
% 	Szell2013SR,
% 	Kramer2014PNAS} 
% have become available including
% data on online dating~\cite{
% 	Barraket2008JSociology,
% 	Bingol2012PartnerArxiv,
% 	Norcie2013LNCS,
% 	Zhao2014IS,
% 	Ellison2013}.
% Compared to 
% $N = 10,047$ 
% of \refcite{Buss1989} 
% and
% $N = 17,637$ of \refcite{Schmitt2008},
% which are very big numbers,
% we will investigate mating patterns of a population of 
% $N = 44,253$
% based on
% the data of an online dating site. 


% % use section* for acknowledgment
% \section*{Acknowledgment}


% The authors would like to thank...


% % Can use something like this to put references on a page
% % by themselves when using endfloat and the captionsoff option.
% \ifCLASSOPTIONcaptionsoff
%   \newpage
% \fi



% % trigger a \newpage just before the given reference
% % number - used to balance the columns on the last page
% % adjust value as needed - may need to be readjusted if
% % the document is modified later
% %\IEEEtriggeratref{8}
% % The "triggered" command can be changed if desired:
% %\IEEEtriggercmd{\enlargethispage{-5in}}

% % references section

% % can use a bibliography generated by BibTeX as a .bbl file
% % BibTeX documentation can be easily obtained at:
% % http://mirror.ctan.org/biblio/bibtex/contrib/doc/
% % The IEEEtran BibTeX style support page is at:
% % http://www.michaelshell.org/tex/ieeetran/bibtex/
% %\bibliographystyle{IEEEtran}
% % argument is your BibTeX string definitions and bibliography database(s)
% %\bibliography{IEEEabrv,../bib/paper}

% %%% HB VVV for doi link
% %%http://tex.stackexchange.com/questions/84133/hyperlink-each-bib-entry-to-its-doi-page
% %\def\mybibdoicolor{\color{blue!75!black}} %change color to suit.
% %\newcommand*{\doi}[1]{\href{http://dx.doi.org/\detokenize{#1} {\raggedright\mybibdoicolor{DOI: \detokenize{#1}}}}
% %%% HB AAA for doi link

% \IEEEtriggeratref{32}
% \bibliographystyle{IEEEtran}
% \bibliography{IEEEabrv,bingol-PaperTemplateIEEE}


% %
% %% <OR> manually copy in the resultant .bbl file
% %% set second argument of \begin to the number of references
% %% (used to reserve space for the reference number labels box)
% %\begin{thebibliography}{1}
% %
% %\bibitem{IEEEhowto:kopka}
% %H.~Kopka and P.~W. Daly, \emph{A Guide to \LaTeX}, 3rd~ed.\hskip 1em plus
% %  0.5em minus 0.4em\relax Harlow, England: Addison-Wesley, 1999.
% %
% %\end{thebibliography}

% % biography section
% % 
% % If you have an EPS/PDF photo (graphicx package needed) extra braces are
% % needed around the contents of the optional argument to biography to prevent
% % the LaTeX parser from getting confused when it sees the complicated
% % \includegraphics command within an optional argument. (You could create
% % your own custom macro containing the \includegraphics command to make things
% % simpler here.)
% %\begin{IEEEbiography}[{\includegraphics[width=1in,height=1.25in,clip,keepaspectratio]{mshell}}]{Michael Shell}
% % or if you just want to reserve a space for a photo:

% %\begin{IEEEbiography}{Haluk~O.~Bingol}
% % \begin{IEEEbiographynophoto}{Haluk~O.~Bingol}
% % 	Assoc. Prof. 
% % 	His research interests include 
% % 	social systems, 
% % 	complex systems, 
% % 	and human brain.
% % 	Ph.D. in Computer Engineering, Syracuse University.
% % 	email:~bingol@boun.edu.tr.
% % \end{IEEEbiographynophoto}
% %\end{IEEEbiography}

% % if you will not have a photo at all:
% % \begin{IEEEbiographynophoto}{Omer~Basar}
% % 	Software Manager.
% % 	His research interests include 
% % 	Machine Learning, 
% % 	Artificial Intelligence, 
% % 	Robotics.
% % 	M.S. in Computer Engineering, Bogazici University.
% % 	email:~omerbasar@gmail.com.	
% % \end{IEEEbiographynophoto}

% % insert where needed to balance the two columns on the last page with
% % biographies
% %\newpage

% % You can push biographies down or up by placing
% % a \vfill before or after them. The appropriate
% % use of \vfill depends on what kind of text is
% % on the last page and whether or not the columns
% % are being equalized.

% %\vfill

% % Can be used to pull up biographies so that the bottom of the last one
% % is flush with the other column.
% %\enlargethispage{-5in}



% % that's all folks
% \end{document}


